\chapter{Introdução}\label{cap:introducao}
Este trabalho está dividido em $7$ capítulos. Neste primeiro capítulo introdutório, serão definidos o problema da ordenação e sua importância. No capítulo seguinte, serão apresentados alguns dos conceitos que serão importantes para o entendimento do conteúdo que se segue. Os próximos três capítulos ($3$, $4$ e $5$) apresentarão, ao todo, $14$ algoritmos que resolvem o problema da ordenação, bem como suas implementações, complexidades temporais e espaciais, entre outras características. O capítulo $6$ realizará, então, uma análise empírica a partir dos resultados obtidos com a execução dos algoritmos, à luz de determinadas métricas. Por fim, o último capítulo trará a conclusão do trabalho. 

O objetivo deste trabalho é, portanto, analisar e comparar os resultados obtidos nas análises teórica e empírica realizadas do capítulo $3$ ao $6$, visando explicar quais características melhor justificam o tempo de execução de cada algoritmo e em quais contextos um algoritmo se destaca em relação aos demais.


\section{O problema da ordenação}
O problema da ordenação consiste em reorganizar uma sequência de dados para que seus elementos estejam dispostos em uma ordem específica, geralmente crescente ou decrescente. Em termos formais, o problema pode ser definido da seguinte maneira: dada uma sequência de  $n$  números  $\langle a_1, a_2, ..., a_n \rangle$, produzir uma permutação $\langle a_1', a_2', ..., a_n' \rangle$ da sequência de entrada, tal que $a_1' \le a_2' \le ... \le a_n'$  \cite[p. 5]{Cormen2009}.

A ordenação é uma das operações fundamentais em computação, com aplicações diretas em diversas áreas, como pesquisa, organização de dados, algoritmos de busca e análise estatística. Além disso, é uma etapa essencial em muitos outros algoritmos, como aqueles utilizados em análise de grafos, aprendizado de máquina e simulação de sistemas.

A importância do problema da ordenação também está relacionada à sua presença em problemas cotidianos, como organização de listas de compras, ordenação de arquivos em sistemas operacionais, ordenação de registros em bancos de dados e priorização de tarefas em filas de execução. Além disso, a resolução desse problema proporciona uma base teórica para o estudo de outros problemas mais complexos em computação.

Compreender e resolver eficientemente o problema da ordenação é crucial para otimizar o desempenho de sistemas computacionais. A escolha de um algoritmo adequado para um determinado cenário pode impactar significativamente os requisitos de tempo e espaço, especialmente em aplicações que envolvem grandes volumes de dados. Assim, o estudo da ordenação não só permite avanços na própria área de computação, mas também contribui para soluções mais eficientes em diversas áreas do conhecimento.
