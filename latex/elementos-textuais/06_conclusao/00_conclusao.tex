\chapter{Conclusão}\label{cap:conclusao}
Neste trabalho, foi definido o problema da ordenação e sua importância. Foram apresentados ao todo $14$ algoritmos para este problema, bem como suas implementações, complexidades temporais e espaciais, dentre outras características. Além disso, a principal contribuição deste trabalho foi dada no capítulo $6$, onde foi realizada uma análise empírica dos resultados obtidos ao se executar cada algoritmo para diferentes configurações de entrada. Lá, foi mostrado que a análise teórica sozinha não basta para escolher um algoritmo de ordenação, pois, na prática, o tempo de execução de um algoritmo também depende da constante que não aparece na notação assintótica, da necessidade ou não de memória adicional, do número de comparações e movimentações realizadas e, também, do quão eficiente o algoritmo é em termos de uso do cache do processador.
