%%%%%%%%%%%%%%%%%%%%%%%%%%%%%%%%%%%%%%%%%%%%%%%%%%%%%%%%%%%%%%%%%%%%%%%%%%%%%%%
%%% 				Customizações do abnTeX2 para o IFCE    			    %%%
%%% 		Instituto Federal de Educação, Ciência e Tecnologia do Ceará    %%%
%%% Template disponível em: https://github.com/clodomirneto/IFCETeX2	    %%%
%%% Desenvolvedores do IFCETeX2: Professor Clodomir Silva Lima Neto		    %%%
%%% 							 Professor Marcelo Araújo Lima              %%%
%%%                              Professor Antonio Sergio de Sousa Vieira	%%%
%%% E-mails para contato: clodomir.neto@ifce.edu.br						    %%%
%%% 					  marcelo.alima@ifce.edu.br                         %%%
%%%                       sergio.vieira@ifce.edu.br	                        %%%
%%%%%%%%%%%%%%%%%%%%%%%%%%%%%%%%%%%%%%%%%%%%%%%%%%%%%%%%%%%%%%%%%%%%%%%%%%%%%%%

% Importações de pacotes
\usepackage[utf8]{inputenc}                         % Acentuação direta
\usepackage[T1]{fontenc}                            % Codificação da fonte em 8 bits
\usepackage{graphicx}                               % Inserir figuras
\usepackage{calc}
\usepackage{svg}
\usepackage{amsfonts,amssymb,amsmath}               % Fonte e símbolos matemáticos
\usepackage{mathtools}
\usepackage{cancel}
\usepackage{booktabs}                               % Comandos para tabelas
\usepackage{verbatim}                               % Texto é interpretado como escrito no documento
\usepackage{multirow,array}                         % Múltiplas linhas e colunas em tabelas
\usepackage{indentfirst}                            % Endenta o primeiro parágrafo de cada seção.
\usepackage{listings}                               % Utilizar codigo fonte no documento
\usepackage{xcolor}
\usepackage{microtype}                              % Para melhorias de justificação?
\usepackage[portuguese,ruled,lined]{algorithm2e}    % Escrever algoritmos
\usepackage{amsgen}
\usepackage{lipsum}                                 % Usar a simulação de texto Lorem Ipsum
\usepackage{tocloft}                                % Permite alterar a formatação do Sumário
\usepackage{etoolbox}                               % Usado para alterar a fonte da Section no Sumário
\usepackage{caption}                                % Altera o comportamento da tag caption
\usepackage[alf,abnt-emphasize=bf,bibjustif, recuo=0cm,abnt-etal-cite=3,abnt-etal-list=0,abnt-etal-text=it]{abntex2cite}  % Citações padrão ABNT
\usepackage{mathptmx}                               % Usa a fonte Times New Roman		
\usepackage{lscape}                                 % Permite páginas em modo "paisagem"
\usepackage{pdflscape}
\usepackage{appendix}                               % Gerar o apendice no final do documento
\usepackage{paracol}                                % Criar paragrafos sem identacao
\usepackage{ifcetex2}		                        % Biblioteca com as normas do IFCE para trabalhos academicos
\usepackage{pdfpages}                               % Incluir pdf no documento
\usepackage{textcomp}
\usepackage{listings}
\usepackage{courier}
\usepackage{float}
\usepackage{makecell}

%TEOREMAS

\newtheorem{proposition}{Proposição}[chapter]
\newtheorem{theorem}{Teorema}[chapter]
\newtheorem{lemma}{Lema}[chapter]
\newtheorem{definition}{Definição}[chapter]
\newtheorem{exemplo}{Exemplo}[chapter]
\newtheorem{corollary}{Corolário}[chapter]
\newtheorem{exercicio}{Exercício}[chapter]

%PROVAS

\newenvironment{prova}[1][Prova]{\noindent\textbf{#1.} }{\hfill\rule{0.5em}{0.5em}}
\newenvironment{dem}[1][Demonstra\c c\~ao]{\noindent\textbf{#1.} }{\hfill\rule{0.5em}{0.5em}}
\newenvironment{exm}{\noindent{\textbf{Exemplo:}}}{}

% Alg
\SetAlgorithmName{Pseudocódigo}{Pseudocódigos}{Lista de Pseudocódigos}

\renewcommand{\arraystretch}{1.5}
\setlength{\tabcolsep}{12pt}

\renewcommand{\lstlistlistingname}{Lista de algoritmos}
\renewcommand{\lstlistingname}{Algoritmo}

\definecolor{back}{rgb}{0.,0.,0.}
\definecolor{codegreen}{rgb}{0,0.6,0}
\definecolor{codegray}{rgb}{0.5,0.5,0.5}
\definecolor{codepurple}{rgb}{0.58,0,0.82}
\definecolor{backcolour}{rgb}{0.95,0.95,0.92}
\definecolor{backcolorgray}{rgb}{0.95703125,0.95703125,0.95703125}

\lstdefinestyle{mystyle}{
    commentstyle=\color{codegray},
    keywordstyle=\bfseries,
    basicstyle=\ttfamily\footnotesize,
    breaklines=true,
    numbers=left,
    frame=none,
    framexleftmargin=2pt,
}

\lstset{style=mystyle}

\lstset{literate=
  {á}{{\'a}}1 {é}{{\'e}}1 {í}{{\'i}}1 {ó}{{\'o}}1 {ú}{{\'u}}1
  {Á}{{\'A}}1 {É}{{\'E}}1 {Í}{{\'I}}1 {Ó}{{\'O}}1 {Ú}{{\'U}}1
  {ã}{{\~a}}1 {õ}{{\~o}}1 {ç}{{\c{c}}}1 {Ç}{{\c{C}}}1
}

% \lstset{basicstyle=\footnotesize\ttfamily,breaklines=true}
% \lstset{framextopmargin=50pt,frame=shadowbox}
% \lstset{framextopmargin=50pt,frame=bottomline}

\let\oldlstlistoflistings\lstlistoflistings
\renewcommand{\lstlistoflistings}{%
  \begingroup
  \let\oldaddcontentsline\addcontentsline % Save the original \addcontentsline
  \renewcommand{\addcontentsline}[3]{} % Redefine \addcontentsline to do nothing
  \oldlstlistoflistings % Call the original \lstlistoflistings
  \endgroup
}
\newcommand{\imprimirlistadecodigos}{
	\lstlistoflistings
	\pagebreak %
}

\newenvironment{courierfont}{\fontfamily{pcr}\selectfont}{\par}

% Forest

\usepackage[edges]{forest}

\definecolor{foldercolor}{RGB}{124,166,198}

\tikzset{pics/folder/.style={code={%
    \node[inner sep=0pt, minimum size=#1](-foldericon){};
    \node[folder style, inner sep=0pt, minimum width=0.3*#1, minimum height=0.6*#1, above right, xshift=0.05*#1] at (-foldericon.west){};
    \node[folder style, inner sep=0pt, minimum size=#1] at (-foldericon.center){};}
    },
    pics/folder/.default={20pt},
    folder style/.style={draw=foldercolor!80!black,top color=foldercolor!40,bottom color=foldercolor}
}

\forestset{is file/.style={edge path'/.expanded={%
        ([xshift=\forestregister{folder indent}]!u.parent anchor) |- (.child anchor)},
        inner sep=1pt},
    this folder size/.style={edge path'/.expanded={%
        ([xshift=\forestregister{folder indent}]!u.parent anchor) |- (.child anchor) pic[solid]{folder=#1}}, inner xsep=0.7*#1},
    folder tree indent/.style={before computing xy={l=#1}},
    folder icons/.style={folder, this folder size=#1, folder tree indent=3*#1},
    folder icons/.default={12pt},
}

% resize sum [https://tex.stackexchange.com/a/81711]
\newlength{\depthofsumsign}
\setlength{\depthofsumsign}{\depthof{$\sum$}}
\newlength{\totalheightofsumsign}
\newlength{\heightanddepthofargument}

\newcommand{\nsum}[1][1.4]{% only for \displaystyle
    \mathop{%
        \raisebox
            {-#1\depthofsumsign+1\depthofsumsign}
            {\scalebox
                {#1}
                {$\displaystyle\sum$}%
            }
    }
}
\newcommand{\resum}[1]{%
    \def\s{#1}
    \mathop{
        \mathpalette\resumaux{#1}
    }
}

\newcommand{\resumaux}[2]{% internally
    \sbox0{$#1#2$}
    \sbox1{$#1\sum$}
    \setlength{\heightanddepthofargument}{\wd0+\dp0}
    \setlength{\totalheightofsumsign}{\wd1+\dp1}
    \def\quot{\DivideLengths{\heightanddepthofargument}{\totalheightofsumsign}}
    \nsum[\quot]%
}

% http://tex.stackexchange.com/a/6424/16595
\makeatletter
\newcommand*{\DivideLengths}[2]{%
  \strip@pt\dimexpr\number\numexpr\number\dimexpr#1\relax*65536/\number\dimexpr#2\relax\relax sp\relax
}
\makeatother

% Big O notation
% \newcommand\bigO[1]{$\mathcal{O}(#1)$}
\newcommand\bigO[1]{$O(#1)$}
